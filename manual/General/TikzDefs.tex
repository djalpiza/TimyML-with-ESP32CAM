%%%%%%%%%%%%%%%%%%%%%%%%%%%%%%
%
% $Autor: Wings $
% $Datum: 2020-07-24 09:05:07Z $
% $Pfad: GDV/Vortraege/latex - Ausarbeitung/Allgemein/tikzdefs.tex $
% $Version: 4732 $
%
%
%%%%%%%%%%%%%%%%%%%%%%%%%%%%%%


% Definitionen für tikz

\usepackage{pgfplots}
\usepackage{pgf,tikz}
\usepackage{mathrsfs}

\usetikzlibrary{shapes,shapes.symbols,shapes.misc, shapes.multipart, shapes.geometric,arrows,angles,quotes,babel,positioning, calc,math,
backgrounds}


\tikzset{
  input2/.style={ % requires library shapes.geometric
    draw,
    trapezium,
    trapezium left angle=60,
    trapezium right angle=120,
  },
  process rectangle outer width/.initial=0.15cm,
  predefined process/.style={
    rectangle,
    draw,
    append after command={
      \pgfextra{
        \draw [fill=blue!20]
        ($(\tikzlastnode.north west)-(0,0.5\pgflinewidth)$)--
        ($(\tikzlastnode.north west)-(\pgfkeysvalueof{/tikz/process rectangle outer width},0.5\pgflinewidth)$)--
        ($(\tikzlastnode.south west)+(-\pgfkeysvalueof{/tikz/process rectangle outer width},+0.5\pgflinewidth)$)--
        ($(\tikzlastnode.south west)+(0,0.5\pgflinewidth)$);
        \draw [fill=blue!20]
        ($(\tikzlastnode.north east)-(0,0.5\pgflinewidth)$)--
        ($(\tikzlastnode.north east)+(\pgfkeysvalueof{/tikz/process rectangle outer width},-0.5\pgflinewidth)$)--
        ($(\tikzlastnode.south east)+(\pgfkeysvalueof{/tikz/process rectangle outer width},0.5\pgflinewidth)$)--
        ($(\tikzlastnode.south east)+(0,0.5\pgflinewidth)$);
      }  
    },
    text width=#1,
    align=center, fill=blue!20,  minimum height=4em
  },
  predefined process/.default=20mm,
  data1/.style={
    trapezium, 
    trapezium left angle=70, 
    trapezium right angle=110, 
    text width=1.5cm, 
    inner ysep=17pt,
    align=center, 
    line width=2pt,
    fill=blue!20
  },      
}


% Parameter
% #1: Skalierung
% #2: Winkel; 0..179
\newcommand{\HermiteSym}[2]{%
   \pgfmathsetmacro{\RADIUS}{6}
  
   \begin{scope}[scale=#1]

     % angle 
     \begin{scope}[shift={(\RADIUS,0cm)}]
       \draw[fill=green!30] (0,0) -- (180:0.25*\RADIUS) arc (180:#2:0.25*\RADIUS);
       \draw ({0.5*(180+#2)}:{0.175*\RADIUS}) node {$\beta$};
       \draw ({0.5*(#2)}:{0.175*\RADIUS}) node {$\alpha$}; %$\pi-\alpha$
     \end{scope}

     \coordinate[label=left:$P_0$]  (P0) at (0,0);
     \coordinate  (t0) at (0.25*\RADIUS,0);
     \coordinate[label=below:$S$]  (S) at (\RADIUS,0);
     \coordinate  (s0) at (1.3*\RADIUS,0);
     \coordinate[label=left:$P_1$] (P1) at ({\RADIUS+\RADIUS*cos(#2)},{\RADIUS*sin(#2)});
     \coordinate (t1) at ({\RADIUS+ 1.25*\RADIUS*cos(#2)},{1.25*\RADIUS*sin(#2)});
  
     \coordinate[label=below:$\vec{t}_0$](T0) at ($ (P0)!.5!(t0) $);
     \coordinate[label=right:$\vec{t}_1$](T1) at ($ (P1)!.5!(t1) $);

     \draw [black,line width=0.5pt,domain=0:#2,->] plot ({\RADIUS+0.25*\RADIUS*cos(\x)}, {0+0.25*\RADIUS*sin(\x)});

     \draw [line width=1.5pt] (P0) -- (S) --(P1);
     \draw [line width=2pt,->,color=red] (P0) -- (t0);
     \draw [line width=2pt,->,color=red] (P1) -- (t1);
     \draw [line width=0.2pt,dotted] (S) --(s0);
     \node (P00) at (P0) {$\bullet$};
     \node (P11) at (P1) {$\bullet$};
  \end{scope}
}


% Parameter
% #1: Skalierung
% #2: Winkel; 0..-179
\newcommand{\HermiteSymNeg}[2]{%
  \pgfmathsetmacro{\RADIUS}{6}
  
  \begin{scope}[scale=#1]
    
    % angle 
    \begin{scope}[shift={(\RADIUS,0cm)}]
      \draw[fill=green!30] (0,0) -- (-180:0.25*\RADIUS) arc (-180:#2:0.25*\RADIUS);
      \draw ({0.5*(-180+#2)}:{0.175*\RADIUS}) node {$\beta$};
      \draw ({0.5*(#2)}:{0.175*\RADIUS}) node {$\alpha$}; %$\pi-\alpha$
    \end{scope}
    
    \coordinate[label=left:$P_0$]  (P0) at (0,0);
    \coordinate  (t0) at (0.25*\RADIUS,0);
    \coordinate[label=above:$S$]  (S) at (\RADIUS,0);
    \coordinate  (s0) at (1.3*\RADIUS,0);
    \coordinate[label=left:$P_1$] (P1) at ({\RADIUS+\RADIUS*cos(#2)},{\RADIUS*sin(#2)});
    \coordinate (t1) at ({\RADIUS+ 1.25*\RADIUS*cos(#2)},{1.25*\RADIUS*sin(#2)});
    
    \coordinate[label=below:$\vec{t}_0$](T0) at ($ (P0)!.5!(t0) $);
    \coordinate[label=right:$\vec{t}_1$](T1) at ($ (P1)!.5!(t1) $);
    
    \draw [black,line width=0.5pt,domain=0:#2,->] plot ({\RADIUS+0.25*\RADIUS*cos(\x)}, {0+0.25*\RADIUS*sin(\x)});
    
    \draw [line width=1.5pt] (P0) -- (S) --(P1);
    \draw [line width=2pt,->,color=red] (P0) -- (t0);
    \draw [line width=2pt,->,color=red] (P1) -- (t1);
    \draw [line width=0.2pt,dotted] (S) --(s0);
    \node (P00) at (P0) {$\bullet$};
    \node (P11) at (P1) {$\bullet$};
  \end{scope}
}

% Beispiele
%
%\begin{center}
%  \begin{tikzpicture}
%  \HermiteSym{1}{120}
%  \end{tikzpicture}
%\end{center}
%
%\begin{center}
%  \begin{tikzpicture}
%  \HermiteSym{1}{70}
%  \end{tikzpicture}
%\end{center}
%
%\begin{center}
%  \begin{tikzpicture}
%  \HermiteSym{1}{20}
%  \end{tikzpicture}
%\end{center}
%
%\begin{center}
%  \begin{tikzpicture}
%  \HermiteSym{1}{0}
%  \end{tikzpicture}
%\end{center}
%
%
%\begin{center}
%  \begin{tikzpicture}
%  \HermiteSymNeg{1}{-20}
%  \end{tikzpicture}
%\end{center}
%
%\begin{center}
%  \begin{tikzpicture}
%  \HermiteSymNeg{1}{-70}
%  \end{tikzpicture}
%\end{center}
%
%
%\begin{center}
%  \begin{tikzpicture}
%  \HermiteSymNeg{1}{-120}
%  \end{tikzpicture}
%\end{center}
%
