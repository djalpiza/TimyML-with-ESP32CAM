%%%%%%
%
%%%
%
% $Autor: Wings $
% $Datum: 2021-05-14 $
% $Pfad: GitLab/210419KDDInverseKinematic $
% $Dateiname: Introduction
% $Version: 4620 $
%
% !TeX spellcheck = de_GB
% !TeX program = pdflatex
% !BIB program = biber
% !TeX encoding = utf8
%
%%%


\chapter{Maintenance and Troubleshooting}


\section{Maintenance}
Maintenance for a digital meter created using an ESP32 would include regular checks of the sensors to ensure they are working properly, and replacing any sensors that are not working.\\

Ensure that the ESP32 cam and the meter surfce is dust free to ensure correct and accurate readings.\\

Also, to make sure that ESP32 cam can read the meter correctly, proper lighting and temperature conditions should be maintained.\\

\section{Troubleshooting}
Troubleshooting for a digital meter created using an ESP32 would involve checking for common issues such as incorrect wiring or programming errors.\\

Firstly it should be identified whether the issue is related to hardware or software. If it is a hardware issue then the device would need to be replaced and the model would need to be transferred to the new hardware device.\\

Troubleshooting the software issue on ESP32 would require connecting it to the laptop/desktop to debug and reinstall the model code.\\

\subsection{Steps to troubleshoot}
\begin{itemize}
	\item Connect the ESP32 to the PC: Attach the ESP32-CAM to the ESP32-MB
	\item Connect the ESP32-MB with the USB cable to the PC/laptop USB port.
	\item Install the necessary drivers: Check if your PC has the necessary drivers installed to recognize the ESP32. If not, you can download the latest drivers from the official website of the ESP32 manufacturer.
	\item Open the Arduino IDE: Launch the ESP-IDF on your PC and make sure that it recognizes the ESP32.
	\item Upload code to the ESP32: Use the upload button in the ESP-IDF to upload code to the ESP32. Make sure that you have selected the correct board and port in the tools menu before uploading.
	\item Open the serial monitor in the ESP-IDF to check for any error messages or issues. This will help you troubleshoot any issues with the code or the connection between the ESP32 and the PC.
	\item Check the connections: Make sure that all the connections between the ESP32 and the PC are secure, and that the connections between the ESP32 and other devices are also secure.
	\item Try a different USB cable or USB port: Sometimes, the issue may be with the USB cable or the USB port. Try using a different cable or a different USB port on your PC.
	\item Check the board settings: Make sure that the board settings in the Arduino IDE match the ESP32 board you are using.
	\item Check the power supply: Make sure that the ESP32 is receiving power. It could be that the USB cable is not providing enough power, and you may need to use an external power supply.
	\item Check online resources: If you are still unable to connect your ESP32 to your PC, check online resources such as forums, documentation, or FAQs for troubleshooting tips and solutions.
\end{itemize}

\section{Safety Precautions}
\begin{itemize}
	\item Store the ESP32 device in a 3D printed case.
	\item Avoid touching the circuit board and wires with bare hands to avoid electric shocks.
	\item For proper grounding make sure that the ESP32 is properly grounded to avoid damage from voltage transients and other electrical disturbances.
	\item Observe ESD protection: Electrostatic discharge (ESD) can damage the ESP32, so it is important to observe ESD protection when handling or working with the device.
\end{itemize}

\section{FAQs}
For frequently asked questions, please visit
\href{https://docs.espressif.com/_/downloads/espressif-esp-faq/en/latest/pdf/}{ESP-FAQ}

