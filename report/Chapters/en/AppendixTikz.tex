% !TeX document-id = {9ccc6fb7-c5f4-4308-9c17-4c226ad62230}
%%%%%%
%
% $Autor: Alpizar, Kumari, JK $
% $Datum: 2023-01-31 11:59:00Z $
% $Version: 1.0.0 $
%
%
% !TeX encoding = utf8
% !TeX root = Rename
% !TeX TXS-program:bibliography = txs:///biber
%
%%%%%%
%%%%%%

\chapter{drawings with tikz}


The package tikz is a powerful tool for creating graphics. Many introductions exist. Here only the first steps are shown, so that you can easily create flowcharts.

\bigskip

Drawing a line and arrows

\lstinputlisting{tikz/Line.tex}

\medskip

\input{tikz/Line.tex}

\bigskip

Drawing a thick blue line

\lstinputlisting{tikz/Line2.tex}

\medskip

\input{tikz/Line2.tex}

\bigskip

Drawing an arc

\lstinputlisting{tikz/arc.tex}

\medskip

\input{tikz/arc.tex}

\bigskip

Draw a function

\lstinputlisting{tikz/function.tex}

\medskip

\input{tikz/function.tex}

\bigskip

Drawing rectangles and moving objects

\lstinputlisting{tikz/rectangle.tex}

\medskip

\input{tikz/rectangle.tex}

\bigskip

%input text left right bottom top

\bigskip

Use of variables

\lstinputlisting{tikz/variable.tex}

\medskip

\input{tikz/variable.tex}

\bigskip

Use of points

\lstinputlisting{tikz/node.tex}

\medskip

\input{tikz/node.tex}

\bigskip

Use of nodes

\lstinputlisting{tikz/node3.tex}

\medskip

\input{tikz/node3.tex}

\bigskip

Use of nodes

\lstinputlisting{tikz/node2.tex}

\medskip

\input{tikz/node2.tex}
